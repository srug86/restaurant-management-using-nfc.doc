\chapter{Acceso a bases de datos \texttt{MySQL} desde \texttt{.NET}}
\label{chap:mySQL}

El presente anexo explica los pasos a seguir para realizar conexiones a una
base de datos \texttt{MySQL}, a través de una aplicación implementada en
\texttt{.NET}, utilizando la biblioteca de enlace dinámico
\emph{MySql.Data.dll}:

\begin{enumerate}
\item En primer lugar, se crea una nueva instancia de una conexión
\emph{MySQL}. Para ello, se utiliza una sentencia similar a esta:

\begin{lstlisting}[
  language = C]
  MySqlConnection conexion =
    new MySqlConnection("server='svr';database='db';uid='user';pwd='pass'");
\end{lstlisting}

Donde \texttt{'svr'} es la dirección \acs{URL} del servidor que aloja a la base
de datos, \texttt{'db'} es el nombre de la base de datos y \texttt{'user'} y
\texttt{'pass'} son el nombre y la contraseña respectivamente del usuario
con el que se va a acceder.

\item Antes de realizar cualquier consulta, es necesario
abrir una conexión activa con la base de datos. Para ello, se utiliza el
método \texttt{Open()}, que forma parte de la clase \texttt{MySqlConection}:

\begin{lstlisting}[
  language = C]
  conexion.Open();
\end{lstlisting}

\item A continuación, se construye un comando de tipo \texttt{MySqlCommand}
utilizando para ello una sentencia \acs{SQL}, con la consulta que se desea
realizar y la instancia del objeto \texttt{MySqlConection} previamente
abierta:

\begin{lstlisting}[
  language = C]
  MySqlCommand comando = new MySqlCommand(sentencia, conexion);
\end{lstlisting}

\item Para ejecutar el comando construido se utilizará uno de estos dos métodos
de la clase \texttt{MySqlCommand}, dependiendo del tipo de consulta a realizar:
\begin{itemize}
\item Si la consulta va a devolver un resultado, es decir, si la consulta
es del tipo \texttt{SELECT}, se utilizará el método \texttt{ExecuteReader()}
y el resultado será almacenado en una variable de tipo
\texttt{MySqlDataReader}:

\begin{lstlisting}[
  language = C]
  MySqlDataReader resultado = comando.ExecuteReader();
\end{lstlisting}

\item Por otro lado, si la consulta no va a devolver ningún resultado, es
decir, si la consulta es del tipo: \texttt{INSERT}, \texttt{UPDATE},
\texttt{DELETE}, \texttt{TRUNCATE}, etc.; se utilizará el método
\texttt{ExecuteNonQuery()}:

\begin{lstlisting}[
  language = C]
  comando.ExecuteNonQuery();
\end{lstlisting}

La consulta tendrá efecto en la base de datos pero no devolverá ningún
resultado.
\end{itemize}

\item Por último, después de realizar una consulta, es preciso cerrar la
conexión abierta. Para ello se hace uso del método \texttt{Close()}, que forma
parte (al igual que \texttt{Open()}) de la clase \texttt{MySqlConection}:

\begin{lstlisting}[
  language = C]
  conexion.Close();
\end{lstlisting}
\end{enumerate}

El usuario con el que la aplicación de los servicios web accede a la base
de datos, tendrá privilegios para realizar consultas \acs{SQL} que afecten
a los datos (\texttt{SELECT}, \texttt{INSERT}, \texttt{UPDATE},
\texttt{DELETE} o \texttt{TRUNCATE}), pero no tendrá acceso a operaciones
que afecten a la estructura  (\texttt{CREATE}, \texttt{DROP}, \texttt{TRIGGER},
etc.) o a la administración (\texttt{CREATE USER}, \texttt{SHUTDOWN},
\texttt{RELOAD}, etc.) de la base de datos.

