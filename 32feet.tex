\chapter{Creación de un servidor \texttt{Bluetooth} con \texttt{32feet.NET}}
\label{chap:32feet}

Según~\cite{bib:32feet}, \texttt{32feet.NET}: \emph{``es un proyecto de código 
fuente compartido que tiene como objetivo hacer que tecnologías \acs{PAN}, como 
\texttt{Bluetooth}, \acs{IrDA}, etc., sean más fácilmente accesibles desde 
\texttt{.NET}. Es compatible con sistemas de escritorio, móviles o incrustados.
Actualmente, el proyecto se compone de las tecnologías: \texttt{Bluetooth},
\acs{IrDA} y \acs{OBEX}.''}.

La implementación de un servidor \texttt{Bluetooth} que publica un servicio,
utilizando la biblioteca \texttt{32feet.NET}, es muy sencilla:
\begin{itemize}
\item Antes de iniciar el servidor, es necesario activar el \texttt{Bluetooth}.
Para ello, se crea una instancia de la clase \texttt{BluetoothRadio} con
el valor \texttt{PrimaryRadio}, que es la \emph{BluetoothRadio} estándar de
esta biblioteca. Y después, se establece el modo de dicha instancia como
\texttt{Discoverable}, para que los clientes puedan detectarla.

  \begin{lstlisting}[
    language = C]
    BluetoothRadio br = BluetoothRadio.PrimaryRadio;
    br.Mode = RadioMode.Discoverable;
  \end{lstlisting}

\item A continuación, se crea una instancia del objeto
\texttt{BluetoothListener}, que se encargará de escuchar las conexiones de los
clientes. El método constructor de este tipo de objetos requiere un atributo
de tipo \texttt{Guid}\footnote{Representa un identificador exclusivo global y 
consiste en un entero de 128 bits (16 bytes) que se puede utilizar en todos los 
equipos y redes siempre que se requiera un identificador único. Es muy 
improbable que este identificador se duplique.}. Este atributo representa al 
identificador del servicio que el servidor \texttt{Bluetooth} va a publicar.

  \begin{lstlisting}[
    language = C]
    Guid servicio = 
      new Guid("{01234567-89AB-CDEF-0123-456789ABCDEF}");
    BluetoothListener btListener = new BluetoothListener(servicio);
  \end{lstlisting}

Los clientes que quieran utilizar este servicio, tendrán que conocer cuál
es el valor de este identificador.

\item Para arrancar el servidor \emph{Bluetooh} con el servicio especificado
se ejecuta el método \texttt{Start()}:

  \begin{lstlisting}[
    language = C]
    btListener.Start();
  \end{lstlisting}

En este momento, el servicio del servidor será visible a los clientes.

\item Para atender las peticiones de los clientes se cuenta con el método
\texttt{AcceptBluetoothClient()}. Este método va a permitir crear objetos
de tipo \texttt{BluetoothClient} con los que gestionar las comunicaciones
de entrada y salida con el cliente. El listado \ref{code:btListener}
muestra un método que implementa un servicio de recepción de archivos de
los clientes:

\begin{lstlisting}[
  language = C,
  caption = {Método que implementa un servicio de recepción de archivos de
  los clientes.},
  label = code:btListener]
  public void servicio()
  {
    string fichero, buffer;
    int count;
    do
    {
      try
      {
        //esperamos a un nuevo cliente Bluetooth
        BluetoothClient cliente = btListener.AcceptBluetoothClient();
        //obtenemos un stream del cliente para poder leer los datos que nos envie
        StreamReader streamR =
          new StreamReader(cliente.GetStream(), Encoding.UTF8);
        //leemos el nombre del fichero
        fichero = streamR.ReadLine();
        if (fichero != "")
        {
          //creamos un nuevo fichero con ese nombre
          FileStream streamW = new FileStream(fichero,
          FileMode.OpenOrCreate, FileAccess.Write);
          BinaryWriter bw = new BinaryWriter(streamW);
          do
          {
            //leemos una linea enviada por el cliente
            buffer = streamR.ReadLine();
            //la procesamos
            byte[] data = Convert.FromBase64String(buffer);
            count = data.Length;
            //la escribimos en el fichero
            if (count > 0)
              bw.Write(data, 0, count);
            //hasta que detecte el final del fichero
          } while (!streamR.EndOfStream);
          MessageBox.Show("Fichero recibido: " + fichero);
          //cerramos el fichero
          bw.Flush();
          bw.Close();
        }
        //cerramos todas la conexion con el cliente
        streamR.Close();
        cliente.Close();
      }
      catch (Exception ex)
      {
        if (!salir)
          MessageBox.Show("Error: " + ex.Message);
      }
    } while (!salir);
  }
\end{lstlisting}

El método \texttt{Close()}, de la clase \texttt{BluetoothClient}, finaliza
la conexión con el cliente.

\item Por último, para cerrar el servidor \emph{Bluetooh} activo
se hará uso del método \texttt{Stop()}:

  \begin{lstlisting}[
    language = C]
    btListener.Stop();
  \end{lstlisting}

\end{itemize}

