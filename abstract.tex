% -*- coding: utf-8 -*-

\chapter{Resumen}
En la actualidad, el uso de dispositivos móviles se ha convertido en una tarea
tan común que, mucha gente ya realiza gran parte de sus gestiones cotidianas a
través de uno de estos dispositivos. Y es que, de un tiempo a esta parte, los
fabricantes de dispositivos móviles dejaron de preocuparse
por seguir disminuyendo sus tamaños y empezaron a incorporar tecnologías que,
hasta hace no mucho, parecían futuristas: desde inclusión de juegos, 
reproducción de música MP3, \acs{SMS}, agenda electrónica o fotografía y vídeo 
digital; a videollamadas, navegación por Internet, \acs{GPS} o televisión 
digital.

Hoy en día, las operadoras y los fabricantes siguen buscando nuevos e
innovadores servicios que atraigan a los clientes a consumir sus 
productos. Una de las últimas tecnologías que han sido incorporadas en algunos
modelos es la tecnología \acs{NFC}. \acs{NFC} es una
tecnología de comunicación inalámbrica, de corto alcance y alta frecuencia que
permite el intercambio de datos entre dispositivos. Esta tecnología, unida
al potencial que ofrecen los dispositivos móviles actuales, puede utilizarse
para mejorar las interacciones que tiene el usuario con el entorno inteligente
que lo rodea, haciéndolas más intuitivas y rápidas.

En el presente documento se explican los pasos que se han seguido en la
búsqueda por aplicar la tecnología \acs{NFC} (y a otras tecnologías
móviles como \texttt{Java} o \emph{Bluetooth}) para complementar, facilitar y
enriquecer las principales tareas que se llevan a cabo en un restaurante en
la interacción con sus clientes, tales como: la asignación de mesas, la
realización de pedidos o el pago de estos. Además, también se han estudiado
algunas formas de incentivar a los clientes para que empiecen a utilizar esta
tecnología, como son: la aplicación de descuentos o el ofrecimiento de
recomendaciones personalizadas, calculadas a partir de las interacciones
anteriores de un mismo cliente.


% Local Variables:
%   coding: utf-8
%   mode: latex
%   TeX-master: "main"
%   mode: flyspell
%   ispell-local-dictionary: "castellano8"
% End:
