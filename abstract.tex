% -*- coding: utf-8 -*-

\chapter{Resumen}
En la actualidad, el uso de los dispositivos móviles se ha convertido en una
tarea tan común que, mucha gente ya realiza gran parte de sus gestiones 
cotidianas a través de uno de ellos. Y es que, de un tiempo a esta parte, los
fabricantes han dejado de preocuparse por seguir disminuyendo el tamaño de
sus dispositivos y se han centrado en dotarlos de toda clase de tecnologías y 
funcionalidades: desde la inclusión de juegos, reproducción de música MP3,
\acs{SMS}, agenda electrónica o fotografía y vídeo digital; a videollamadas, 
navegación por Internet, \acs{GPS} o televisión digital.

Hoy en día, las operadoras y los fabricantes siguen buscando nuevos e
innovadores servicios que atraigan a más clientes a consumir sus 
productos. Una de las últimas tecnologías que han sido incorporadas en algunos
modelos, es la tecnología \acs{NFC}. Esta tecnología, unida
al potencial que ofrecen los dispositivos móviles actuales, puede utilizarse
para mejorar las interacciones que tiene el usuario con el entorno inteligente
que lo rodea, haciéndolas más intuitivas y rápidas.

El presente proyecto pretende hacer uso de esta tecnología (y de otras
tecnologías móviles como \texttt{Java} o \texttt{Bluetooth}) para desarrollar 
e implementar el prototipo de un sistema que complemente, facilite y enriquezca
la realización de las principales tareas que se llevan a cabo en un
restaurante, en la interacción con sus clientes, como son: la asignación de 
mesas, la realización de pedidos o el pago de estos. Además, se han
desarrollado fórmulas para incentivar a los clientes para que empiecen a 
usar esta tecnología, como son: la aplicación de descuentos y el 
ofrecimiento de recomendaciones personalizadas, calculadas a partir de las 
interacciones anteriores de un mismo cliente.


% Local Variables:
%   coding: utf-8
%   mode: latex
%   TeX-master: "main"
%   mode: flyspell
%   ispell-local-dictionary: "castellano8"
% End:
