\chapter{Creación de servicios web en \texttt{.NET}}
\label{chap:webServices}

La creación de \emph{servicios web} en \texttt{.NET} es relativamente sencilla
utilizando el  \acs{IDE} \texttt{Visual Studio}:
\begin{enumerate}
\item En primer lugar, se creará un nuevo proyecto \texttt{C\#} del tipo
``\texttt{ASP .NET Web Service}''.
\item A continuación, se añadirá al proyecto un nuevo servicio web. Los
servicios web aparecen en el proyecto como un archivo con extensión
``\texttt{.asmx}''.
\item En el código del fichero ``\texttt{.asmx}'' que se acaba de crear, se
añadirá el atributo \texttt{WebService} para indicar el
\emph{namespace}\footnote{Es una palabra clave de \texttt{.NET} que se utiliza
para declarar un \emph{ámbito}. Este \emph{ámbito} permite organizar el código
y proporciona una forma de crear tipos globalmente únicos. Un mismo
\emph{namespace} puede dar cabida a varios proyectos.} al que pertenecerá el
servicio y una descripción del mismo (listado \ref{code:webService}):

\begin{lstlisting}[
  language = C,
  caption  = {Ejemplo de creación de un servicio web.},
  label    = code:webService]
  [WebService(
    Namespace="http://www.miSitio.es/MiNamespace",
    Description="Aqui va la descripcion de mi servicio web")]
  public class MiServicioWeb : System.Web.Services.WebService {
    ...
  }
\end{lstlisting}

\item El siguiente paso será añadir la definición del servicio web. Por norma
general, y para que el servicio sea interoperable al 100\, es conveniente
incluir una sóla operación por cada servicio web, sin embargo, \texttt{.NET}
permite incluir todas las operaciones que se necesiten. Para ello, sólo es
necesario aplicar el atributo \emph{WebMethod} al método que se quiera publicar
(listado \ref{code:webMethod}). Si no se añade dicho atributo, el método no
será visible desde el servicio web. Al igual que con el atributo
\emph{WebService}, \emph{WebMethod} también admite una descripción de la
operación.

\begin{lstlisting}[
  language = C,
  caption  = {Ejemplo de creación de un método web.},
  label    = code:webMethod]
  [WebMethod(
    Description="Aqui va la descripcion del metodo web HolaMundo.")]
  public string HolaMundo() {
    return "Hola mundo";
  }
\end{lstlisting}

\item Por último, se compila el proyecto y se publica la aplicación. Se
generará un archivo \acs{WSDL}, que contiene toda la documentación que se
indicó en los atributos \emph{WebService} y \emph{WebMethod} y será el que se 
proporcione a las aplicaciones interesadas en consumir alguno de los servicios
definidos, para que conozcan los parámetros a utilizar y los tipos de retorno.
El contenido del archivo \acs{WSDL} puede consultarse a través de la dirección:
\url{http://www.miSitioWeb.es/MiServicioWeb.asmx?wsdl}.
\end{enumerate}

El siguiente paso sería conectar las aplicaciones de escritorio con el
servicio web creado:
\begin{enumerate}
\item Uno de los atributos del proyecto de la aplicación de escritorio
(desarrollada en \texttt{Visual C\#}, con \emph{Visual Studio}) es
\emph{References}. Este atributo permite asignar referencias a librerías
externas almacenadas en disco o a servicios web. En este caso, se añadirá una
nueva \emph{Web Reference}.
\item A continuación, se ingresa la \acs{URL} donde se encuentra el \acs{WSDL}
creado, es decir: \url{http://www.miSitioWeb.es/MiServicioWeb.asmx?wsdl} y se
le proporciona un nombre a la referencia, por ejemplo \emph{MiReferencia}.
\emph{Visual Studio} genera automáticamente los \emph{stubs}\footnote{Son los
trozos de código que simulan el comportamiento de un código ya existente (pero
en una máquina remota) o sustituyen temporalmente a un código que aún no ha
sido implementado.} del cliente.
\item Por último, para hacer uso del método web definido en el servicio
web, se añadirá un código similar al siguiente (listado
\ref{code:miReferencia}):

\begin{lstlisting}[
  language = C,
  caption  = {Ejemplo de invocación de un método web.},
  label    = code:miReferencia]
  //Primero se crea una instancia del servicio
  MiReferencia.MiServicioWeb ws = new MiReferencia.MiServicioWeb();
  //Ahora ya pueden utilizarse sus metodos
  Console.WriteLine(ws.HolaMundo());
\end{lstlisting}

\end{enumerate}

