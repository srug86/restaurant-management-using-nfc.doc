% -*- coding: utf-8 -*-

\chapter{Estado del arte y trabajos relacionados}
En el siguiente capítulo se realiza una presentación de los conceptos y
tecnologías que han sido utilizadas para la elaboración del proyecto y se hace
un repaso de algunos de los trabajos y aplicaciones reales que han servido de
inspiración para el diseño final del sistema.

\section{Estado del arte}
A continuación, se citan los principales elementos que constituyen la base
teórica sobre la que se sustenta el resto del escrito.

  \subsection{La tecnología \acs{RFID}}
\acs{RFID} son las siglas de \emph{Radio Frequency IDentification} (en
castellano, \emph{\textbf{identificación por radiofrecuencia}}). Se trata de un
sistema de almacenamiento y recuperación de datos remoto que utiliza las ondas
de radio para transmitir la identidad (única) de un objeto.

El modo de funcionamiento de los sistemas \acs{RFID} es simple. Una
\textbf{etiqueta \acs{RFID}} (o transponedor) que contiene datos genera una
señal de radiofrecuencia con dichos datos. Esta señal es captada por un
\textbf{lector \acs{RFID}} y este la transforma a una señal digital entendible
por una aplicación específica que utilice \acs{RFID} (figura
\ref{fig:rfidSystem}).

\begin{figure}[!h]
  \begin{center}
    \includegraphics[width=0.8\textwidth]{rfidSystem.png}
    \caption{Esquema básico de un sistema \acs{RFID}.}
    \label{fig:rfidSystem}
  \end{center}
\end{figure}

    \subsubsection{Historia}
  El origen del \acs{RFID} está relacionado con la II Guerra Mundial. Los
  radares de la época eran capaces de detectar y medir la presencia de objetos
  dentro de un rango de actuación, pero no eran capaces de identificar qué
  clases de objetos eran identificados.

  Durante la década de los 50 y los 60, científicos de los países más
  avanzados, trabajaron para explicar cómo podrían identificar objetos
  remotamente. Fruto de estos estudios se inventaron los primeros sistemas
  antirrobo que funcionaban con ondas de radio. El objeto en cuestión tenía 
  una etiqueta con un único bit que decía si el artículo se había pagado o no.
  Cuando el objeto había sido pagado, se modificaba dicho bit, para los 
  sensores de la salida no accionaran la alarma.

  Las primeras patentes fueron solicitadas en Estados Unidos en 1973. Mario W.
  Cardullo presentó una etiqueta \acs{RFID} activa que portaba una memoria
  rescribible. Y ese mismo año, Charles Walton recibió la patente para un
  sistema \ac{RFID} pasivo, consistente en una tarjeta con un transponedor que
  comunicaba una señal a un lector situado en una puerta. Si la tarjeta era
  validada por el lector, se desbloqueaba la cerradura de la puerta.

  A partir de ese año, la tecnología \acs{RFID} empezó a utilizarse por ejemplo,
  en sistemas de apertura de puertas automáticas en centrales nucleares o en
  sistemas para controlar el ganado que había sido vacunado y el que no.

  En la década de los 90, el desarrollo de nuevos materiales permitió
  reducir drásticamente el precio de las etiquetas. Este hecho favoreció
  que se potenciara el número de aplicaciones que utilizan esta tecnología.
  Es por ello que organismos internacionales empezaran a poner sus esfuerzos en
  desarrollar estándares en el uso de este tipo de etiquetas.

    \subsubsection{Etiquetas \acs{RFID}. Arquitectura y funcionamiento}
  Las etiquetas \acs{RFID} son dispositivos pequeños, similares a una pegatina,
  que pueden incorporarse a un objeto, un animal o una persona. 
  
  Todas las etiquetas \acs{RFID} tienen en común los siguientes elementos
  (figura \ref{fig:rfidComponents}):

  \begin{figure}[!h]
    \begin{center}
      \includegraphics[width=0.4\textwidth]{rfidComponents.png}
      \caption{Componentes de una tarjeta \acs{RFID}.}
      \label{fig:rfidComponents}
    \end{center}
  \end{figure}

  \begin{itemize}
    \item \textbf{Antena}. Se encarga de recibir las señales emitidas por el
  lector y de enviar la respuesta ante dichas señales.
    \item \textbf{Chip}. Contiene la lógica de operación de la etiqueta y un
  número de identificación único.
    \item \textbf{Memoria}. Está compuesta por una parte de sólo lectura,
  que contiene las instrucciones básicas para el funcionamiento de la etiqueta;
  y por una parte de lectura y escritura, que almacena los datos escritos
  durante una comunicación con el lector.
  \end{itemize}
  
  Por otro lado, las etiquetas \ac{RFID} pueden ser de tres tipos:
  \begin{itemize}
    \item \textbf{Pasivas}. No poseen ninguna fuente autónoma de energía. La
  señal del lector es la que le induce una pequeña cantidad energía suficiente 
  como para generar y transmitir la respuesta. Tienen una fiabilidad y una
  capacidad de almacenamiento muy limitadas (unos pocos KBytes) y su campo de
  cobertura es también muy reducido (hasta 3 metros). Aún así, son las más
  utilizadas debido a su bajo coste.
   \item \textbf{Activas}. Poseen su propia fuente autónoma de energía y la
  utilizan para dar corriente a sus circuitos integrados y para propagar su
  señal al lector. Esto implica que las comunicaciones son más fiables (tienen
  menos errores), pueden transmitir señales más potentes y a mayor distancia
  (hasta 500m) y tienen más capacidad de almacenamiento. Por otro lado, tienen
  un mayor coste por chip y son de mayor tamaño que las etiquetas pasivas. La
  vida útil de sus baterías puede llegar hasta los 10 años.
    \item \textbf{Semipasivas}. Al igual que las etiquetas activas, las
  semipasivas también disponen de una fuente autónoma de energía. Sin embargo,
  estas utilizan la energía principalmente para alimentar el chip, no para
  transmitir la señal. Tienen una fiabilidad comparable a la de las etiquetas
  activas aunque superan su vida útil. Por otro lado, tienen un rango
  operativo comparable a las etiquetas pasivas aunque su respuesta es más
  rápida.
  \end{itemize}

    \subsubsection{Lectores RFID. Funcionamiento}
  Los lectores \acs{RFID} son los encargados de leer o re-escribir la
  información almacenada en las etiquetas.
  El funcionamiento es sencillo. La antena del lector crea un campo magnético
  y cuando este campo entra en contacto con una etiqueta, se produce la
  reacción de esta última, enviando al lector la información contenida.
  El lector decocifica los datos obtenidos y los manda a una tercera entidad 
  para que los interprete (figura \ref{fig:rfidSchema}).

  \begin{figure}[!h]
    \begin{center}
      \includegraphics[width=0.8\textwidth]{rfidSchema.png}
      \caption{El lector y la etiqueta son los principales componentes de todo
sistema \acs{RFID}.}
      \label{fig:rfidSchema}
    \end{center}
  \end{figure}

  Según el número de bobinas que poseen, existen dos tipos de lectores:
  \begin{itemize}
  \item \textbf{Bobina simple}. La misma bobina crea el campo magnético y
  transmite los datos. Son las más simples y baratas que las dobles y tienen
  un alcance muy limitado.
  \item \textbf{Bobina doble}. Una de las bobinas se encarga de crear el
  campo magnético y la otra de transmitir los datos. Son caras pero tienen
  mayores prestaciones que las bobinas simples.
  \end{itemize}
  
  En cuanto a la portabilidad, también existen dos tipos de lectores:
  \begin{itemize}
  \item \textbf{Lectores móviles}. Son lectores autónomos que pueden 
  transportarse a cualquier lugar y que pueden utilizarse con varios fines. Se
  comunican con otros dispositivos a través de conexiones inalámbricas.
  \item \textbf{Lectores fijos}. Son lectores ubicados en un punto fijo y 
  dedicados a un único fin. Tienen mayor rango de actuación que los lectores
  móviles y suelen utilizarse en sistemas de detección y seguimiento de
  personas y animales.
  \end{itemize}

    \subsubsection{La tecnología \emph{MIFARE}}
  \emph{MIFARE} es el estándar de la industria para interfaces de
  tarjetas inteligentes sin contacto y lectores que operan a 13.56MHz.
  Funcionan de acuerdo con el estándar \acs{ISO} 14443\cite{bib:mifare}.
  
  El alcance típico de lectura/escritura de etiquetas \emph{MIFARE} sin
  contacto oscila entre los 2 y los 10 cm; y la capacidad más habitual está
  entre los 1 y los 4KB de memoria \acs{EEPROM}.

  Para que los datos sean leídos o escritos es necesaria una autentificación 
  mútua entre el lector y la etiqueta, ya que el acceso a los mismos está 
  protegidos por una clave de 48 bits. La transmisión de datos por
  radiofrecuencia viaja encriptada.

  En la actualidad \emph{MIFARE} es una marca registrada de \emph{NXP
  Semiconductors} (empresa fundada por \emph{Philips}). Ha vendido más de 5 mil
  millones de tarjetas y etiquetas inteligentes y más de 50 millones de
  componentes de lectores. Ha sido seleccionada seleccionada para la mayoría
  de proyectos importantes con tarjetas inteligentes sin contacto en todo el
  mundo y su cartera de productos incluye soluciones perfectas para la
  recaudación automática de tarifas, tarjetas de fidelización, cobro en 
  carreteras de peaje o gestión de acceso a edificios\cite{bib:urlMifare} 
  (figura \ref{fig:mifareFamily}.

  \begin{figure}[!h]
    \begin{center}
      \includegraphics[width=0.5\textwidth]{mifareFamily.png}
      \caption{Ejemplos de etiquetas MIFARE.}
      \label{fig:mifareFamily}
    \end{center}
  \end{figure}

  \subsection{La tecnología NFC}
  \subsection{La tecnología Bluetooth}
  \subsection{Los servicios web}
  \subsection{El framework .NET}
  \subsection{La tecnología Java}
  \subsection{Los displays táctiles}
  \subsection{La computación móvil}

\section{Trabajos relacionados}
  \subsection{Estudios relacionados}
    \subsubsection{Modelos de comercio utilizando NFC}
  Los servicios que ofrece \acs{NFC} están muy diversificados y extendidos
  por todo el mundo. Algunos de los servicios están orientados a utilizar
  las tarjetas inteligentes como un monedero electrónico sustitutivo del
  pago en efectivo en los medios de transporte público, otros sustituyen
  a los clásicos cupones para premiar la fidelidad de los clientes de un
  establecimiento, otros sirven para solicitar un servicio de entre una
  lista de servicios disponibles (como elegir un producto dentro de una
  lista de productos), etc.

  Con el fin de integrar todos estos servicios como un proceso común de
  comercio electrónico basado en NFC, se ha propuesto un modelo genérico
  para el comercio móvil basado en \acs{NFC} (\acs{NMC})
  \cite{bib:nfcCommerce}. Este modelo caracteriza los problemas y las
  tecnologías que intervienen en cada una de las siguientes fases
  (figura \ref{fig:nmcModel}):

  \begin{figure}[!h]
    \begin{center}
      \includegraphics[width=0.5\textwidth]{nmcModel.png}
      \caption{El modelo \acs{NMC}.}
      \label{fig:nmcModel}
    \end{center}
  \end{figure}

  \begin{itemize}
  \item \textbf{Fase de inicialización}. Para la construcción de un servicio
  de comercio basado en \acs{NFC} se necesitan definir cuatro componentes
  fundamentales:
    \begin{itemize}
    \item \textbf{Servicio de \emph{back-end\cite{En diseño de software el
    \emph{front-end} es la parte del software que interactúa con el o los
    usuarios y el \emph{back-end} es la parte que procesa la entrada desde el
    \emph{front-end}}} basado en \acs{NFC}}. Este
    servicio es el responsable de todo el intercambio de información durante
    el proceso de comercio. Está gestionado por el proveedor de servicios
    basados en \acs{NFC}.
    \item \textbf{Aplicación de terminal genérico \acs{GTA}}. Como el tamaño
    de la memoria del dispositivo móvil es limitada, combiene definir una
    aplicación genérica que permita realizar todas las gestiones comerciales.
    \item \textbf{Funcionalidad de las etiquetas}. Definir los tipos de
    etiquetas que se utilizarán para construir un entorno de servicios con
    \acs{NFC} que siga el estándar \acs{NDEF}. Por ejemplo, un tipo de etiqueta
    para iniciar la aplicación, otro para cada uno de los productos, otro
    para seleccionar descuentos y otro para seleccionar el servicio.
    \item \textbf{Habilitar el servicio \acs{NFC}}. La etiqueta de inicio de
    aplicación se encargará de cargar el \acs{GTA} y de arrancar el servicio de
    \emph{back-end}, para que el usuario pueda usarlos nada más iniciar la
    aplicación.
    \end{itemize}
  \item \textbf{Fase de pedido}. Después de la inicialización el cliente ya
  está preparado para realizar un pedido a través de un catálogo de productos:
    \begin{itemize}
    \item Cada \textbf{catálogo} tiene una o varias etiquetas que representan
    los productos ofrecidos por el servicio. Cuando el dispositivo toca una
    de estas etiquetas, la aplicación puede llamar al servicio \emph{back-end}
    para descargar el contenido (imágenes, videos, etc.) relacionado con el
    contenido del producto al que representa la etiqueta.
    \item Además, cada producto puede tener opciones del tipo \emph{sabor},
    \emph{volumen}, \emph{tamaño}; que pueden ser seleccionables a través
    de otras etiquetas o a través de la pantalla del dispositivo.
    \end{itemize}
  \item \textbf{Fase de pago}. Hay dos métodos de pago disponibles en el
  modelo \acs{NMC}:
    \begin{itemize}
    \item \textbf{Modo monedero electrónico} (figura \ref{fig:e-wallet}). 
    Después de que el cliente haya realizado un pedido, el \acs{GTA} 
    solicitará la información del producto al servicio \emph{back-end}. Si el 
    servicio \emph{back-end} confirma la autenticidad del usuario y de la 
    operación, le devolverá al \acs{GTA} el importe del producto. Esta 
    cantidad será decrementada del monedero electrónico. El monedero 
    electrónico es un elemento seguro al que sólo se puede acceder con el 
    permiso del servicio \emph{back-end}. Los clientes deben disponer de saldo 
    para poder realizar un pedido (método prepago).

    \begin{figure}[!h]
      \begin{center}
        \includegraphics[width=0.5\textwidth]{e-wallet.png}
        \caption{Fase de pago con \emph{modo monedero electrónico}.}
        \label{fig:e-wallet}
      \end{center}
    \end{figure}

    \item \textbf{Modo tarjeta de crédito} (figura \ref{fig:creditCard}). 
    Antes de utilizar este modo para llevar a cabo la transacción, el banco 
    debe facilitar al cliente un \emph{applet\footnote{Componente de una 
    aplicación que se ejecuta en el contexto de otro programa. No puede 
    ejecutarse de manera independiente.}} que tiene que ser cargada en el 
    dispositivo móvil. Una vez realizado el pedido, el \acs{GTA} almacena el 
    número de orden como elemento seguro. Para realizar el pago, el cliente 
    aproxima el dispositivo móvil al lector del comerciante y este lee el 
    número de orden almacenado. El servicio de \emph{back-end} comprueba el 
    importe a pagar y se inicia un intercambio de información entre el lector 
    del banco y el \emph{applet} de la tarjeta de crédito. Por último, el 
    lector del banco obtiene el código de autorización para finalizar la 
    transacción.
  
    \begin{figure}[!h]
      \begin{center}
        \includegraphics[width=0.5\textwidth]{creditCard.png}
        \caption{Fase de pago con \emph{modo tarjeta de crédito}.}
        \label{fig:creditCard}
      \end{center}
    \end{figure}

    \end{itemize}
  \item \textbf{Fase de envío y recogida}. En esta fase, el cliente elige
  la forma y el lugar en que recibe el producto o servicio. Para productos
  físicos elegirá el lugar al que deben llevárselo o el lugar en el que quiere
  recogerlo. Y para productos virtuales el dispositivo donde quiere
  almacenarlo.
  \item \textbf{Fase de gestión de fidelización}. El programa de fidelización
  está diseñado para atraer a un cliente a consumir de nuevo. Los cupones son
  herramientas de fidelización comunes y la tecnología \acs{NFC} hace factible
  el concepto de \emph{cupones virtuales}.
    \begin{itemize}
    \item Los clientes pueden recibir cupones tocando una etiqueta que 
    simbolice la descarga de un cupón.
    \item A través de \acs{NFC}, utilizando el modo comunicación
    \emph{peer-to-peer}, los clientes pueden compartir sus cupones con los
    demás. Esto puede aprovecharse para desarrollar un programa de
    \emph{márketing} atractivo y eficaz.
    \end{itemize}
  \item \textbf{Fase de servicio}. Para solicitar un servicio bastará con
  tocar con el dispositivo móvil una etiqueta de tipo servicio. Las diferentes
  opciones de servicio que estén disponibles serán accesibles a través de
  otras etiquetas de servicio o mediante la selección de ese servicio en la
  pantalla del dispositivo.
  \end{itemize}
  
  Dada su generalidad, el modelo basado en el comercio móvil \acs{NMC},
  establece las líneas maestra a tener en cuenta a la hora de desarrollar
  toda clase de servicios móviles; entre ellos por supuesto, un sistema de 
  gestión de restaurantes basado en \acs{NFC}.

    \subsubsection{Métodos de pago móvil a distancia}
  Hoy en día se pueden distinguir dos tipos de pagos a distancia: los que se
  producen de forma remota, como cuando se realizan compras por internet;
  y los que se realizan dentro del mismo sitio donde se está
  comprando. El primero de ellos goza de una gran popularidad, ya que se
  considera un método fácil, cómodo y bastante seguro; y no requiere de una
  infraestructura especial para ser implementado. El segundo en cambio,
  aunque parece prometedor en el futuro, aún no goza de la misma aceptación, 
  debido en parte a que los establecimientos no están dispuestos a 
  invertir su dinero en adquirir una infraestructura adicional para permitir 
  este tipo de pagos sin ver antes que hacerlo les reporte algún tipo de 
  beneficio.

  En la actualidad el pago a distancia por móvil (sin necesidad de contacto) 
  se puede conseguir mediante uno de estos tres métodos:
  \begin{description}
  \item[SIMpass]. Es una tecnología de interfaz dual que combina la
  tradicional tarjeta \acs{SIM} y la tarjeta \acs{RFID} en una sola tarjeta
  estándar. La interfaz de contacto cumple con la norma \acs{ISO} 7816 y la
  interfaz a distancia cumple con el estándar de la norma \acs{ISO} 14443.
  \emph{SIMpass} trabaja con una frecuencia de 13,56MHz. Como no es una
  frecuencia muy alta, es difícil minimizar el tamaño de antena. Existen dos
  maneras de implementar dicha antena. Una, de bajo coste, es utilizando una 
  antena plana que consta de una bobina y que se le añade a la del teléfono
  móvil. Para la otra habría que reformar parte del hardware del teléfono
  móvil, aunque mejora notablemente la estabilidad y fiabilidad de la
  funcionalidad \acs{RFID}.

  \item[\acs{NFC}]. Como se ha visto en secciones anteriores, \acs{NFC} es una
  tecnología que combina la tecnología \acs{RFID} con las comunicaciones de
  corto alcance. En el sistema \acs{NFC} propuesto se distinguen tres partes 
  principales: el controlador, la antena y una unidad de seguridad. La unidad
  de seguridad puede correr a cargo de la tarjeta \acs{SIM} o de la tarjeta
  \acs{SD}. Lo malo es que, tanto en un caso como en otro, habría que integrar
  un módulo o chip de seguridad que no viene por defecto en estas tarjetas.
  Esto se soluciona utilizando la tecnología \acs{NFC} mejorada (\emph{eNFC}).
  Aunque \emph{eNFC} hace uso de la tarjeta \acs{SIM}, sólo tiene que utilizar
  el pin C6 para poder modificar la memoria \emph{EEPROM}, en vez de tener
  que utilizar dos pines (C4 y C8) que además interferían a la hora de
  realizar otras operaciones.

    \item[Programa RF-SIM]. Combinando el módulo de comunicaciones móviles y
  los micro módulos de radio frecuencia (\emph{RF}) de la tarjeta \acs{SIM}
  normal, la tarjeta \emph{RF-SIM} consigue operar a 2,4GHz (una alta 
  frecuencia que tiene una longitud de onda corta). Esto permite que, a través
  de una antena, el módulo \emph{RF-SIM} pueda comunicarse con otros 
  dispositivos externos a distancias entre los 10 y los 500cm, a pesar de que
  dicho módulo tenga un tamaño muy reducido. \emph{RF-SIM} es un tipo de
  etiqueta activa, por lo que necesita una batería para trabajar.
  \end{description}

  
  % Comparativa

%Método de SIMpass ocupó el C4 y C8 pines, que está reservado para la interfaz de alta velocidad de la tarjeta SIM de gran capacidad. Por lo tanto, tiene un conflicto de la evolución futura de la tarjeta SIM. Desde la perspectiva a largo plazo, el método SIMpass no es apto para el desarrollo. La velocidad de lectura y escritura influye en el tiempo de procesamiento de transacciones. Cuanto mayor sea la velocidad es, el tiempo más corto que se necesita, la mayor seguridad y fiabilidad es. Desde este ángulo, el método de SIMpass realiza un poco peor.
SIMpass tiene una antena simple, que cuesta menos. En consecuencia, es más fácil a corto plazo de promoción. Sin embargo, la estabilidad de antena simple es poco mala. Si el usuario abre la tapa posterior o cambia la batería, la antena se rompería. Por lo tanto, influirá en el uso normal. Teniendo en cuenta las malas influencias, que no sugieren que se utilizará. Debido a la necesidad de reformar SIMpass dispositivo de terminal, el costo de que está cerca el método eNFC. Por lo tanto, el método SIMpass no es recomendable. RF-SIM, de menor costo, podría no ser la solución más adecuada. La distancia de reacción es el factor más significativo en el sistema RFID.
Para el pago móvil, la distancia de reacción más corto, más seguro en la transacción. Hasta cierto punto, la distancia depende de la frecuencia de trabajo. 2,4 GHz, hizo uso de la RF-SIM, no es la más segura, obviamente. 2,4 GHz, una frecuencia pública, apoyo a la tecnología de varios, tales como Bluetooth, Wi-Fi, Zigbee, UWB.
Por lo tanto, la interferencia de RF-SIM que sufre de la que es más. RF-SIM funciona mal en la compatibilidad, ya que las terminales punto de venta más comunes 13,56 adoptar en cambio, en 2.4GHz. RF-SIM funciona en modo de corriente activa. Es decir, el teléfono móvil no puede implementar la función RFID cuando es el poder de falla. Por lo tanto, el método de RF-SIM no es recomendable.
NFC. Entre las soluciones NFC, NFC-SIM tiene un problema en los pines de la tarjeta SIM. Y NFC-SD ha costado más amplia que otras soluciones. Así, eNFC es una solución óptima. Los operadores pueden cooperar con la fabricación mediante la personalización de terminales. Debido a que es bueno para complementar las respectivas ventajas. Además, esta cooperación contribuye a controlar a los operadores.


  % Además...

%NO MORE WAITING ON NFC
%desde hace años, cuando los analistas e ingenieros de los teléfonos habló de sustitución de nuestros bolsillos, se concentraron en una sola tecnología: Near Field Communication. NFC permite a los dispositivos ubicados dentro de uno o dos centímetros el uno del otro para establecer una conexión inalámbrica.
%Sin embargo, ocho años después de Nokia, Philips y Sony fundó el NFC Forum, la tecnología todavía no es común en los teléfonos. En los Estados Unidos, sólo dos modelos, disponibles en la red de Sprint, se han incluido con el hardware NFC para apoyar Google Wallet. (Galaxy de Verizon Wireless de teléfono Nexus tiene un chip NFC, pero la empresa bloqueó el uso de Google Wallet.) Sin duda, habrá más teléfonos con funcionalidad NFC para elegir en el futuro: Más de 100 modelos se han anunciado para 2012, según los analistas de la empresa sueca Berg Insight, más que el doble que los 40 modelos lanzados a nivel mundial en 2011.
%Pero sólo la producción de más teléfonos con NFC no es suficiente. Tiendas todavía tenemos que añadir a los lectores NFC a todos sus cajas registradoras, y los minoristas no quieren gastar dinero en eso hasta que hay una demanda significativa. Es un clásico de la gallina y el huevo de la situación.
%Mientras tanto, el móvil de pago de nueva creación no están esperando a que la NFC. Ellos han descubierto que la mayoría de los teléfonos móviles ya cuentan con las tecnologías disponibles para hacer el trabajo compras.

    \subsubsection{Métodos de fidelización de clientes}


  \subsection{Aplicaciones comerciales}
    \subsubsection{Gestión de operaciones típicas de un restaurante}
    En la actualidad existe una innumerable oferta de aplicaciones que ayudan
    a gestionar las labores típicas de un restaurante: gestionar pedidos,
    generar facturas, realizar la función de caja registradora, \dots
      \begin{itemize}
      \item \emph{\textbf{\href{http://www.softrestaurant.com/}
      {SoftRestaurant}}}. Es un software desarrollado por la empresa mexicana
      \emph{\href{http://www.nationalsoft.com.mx/}{National Soft}}.
      Actualmente va por la versión \emph{8.0}.

      % Descripción general y foto

      A diferencia de otros programas de gestión de restaurantes, permite
      describir las recetas de los productos contemplados en la carta. Es 
      decir, permite listar los ingredientes (con sus cantidades) utilizados
      en la elaboración de cada plato, pudiendo además gestionar el
      \emph{stock} de esos ingredientes.

      \item \emph{\textbf{\href{http://www.estudiolegaspi.com.ar/Salewin.html}
      {SaleYa}}}. Es un software desarrollado por la empresa argentina 
      \emph{\href{http://www.estudiolegaspi.com.ar/}{Estudio Legaspi}}.
      Dispone de una versión limitada de prueba.

        % Descripción general y foto

      Permite editar los objetos (mesas, barra, \dots) que conforman la vista 
      del restaurante, para simular la planta del restaurante real. Además, las
      mesas son representadas con un color u otro según su estado (libre u
      ocupada).

      \item \emph{\textbf{\href{http://www.restbar.com/}{RestBar}}}. Es un
      software desarrollado por la empresa mexicana
      \emph{\href{http://www.ambit.com.mx/}{Ambit Tecnology}}. Actualmente se
      encuentra en la versión \emph{12.03a}.

      % Descripción general y foto

      Dispone de un monitor de comandas (situado en la cocina) que permite
      gestionar el estado de elaboración de los platos.

      \item \emph{\textbf{
      \href{http://www.icg.es/?es/software-programas/restaurantes-hosteleria}
      {FrontRest}}}. Es un software desarrollado por la empresa española
      \emph{\href{http://www.icg.es}{ICG Software}}.

      Dispone de varios módulos que satisfacen las necesidades de información
      del restaurante: aplicaciones móvil para repartos a domicilio, aplicación
      para gestionar los pedidos en la barra, aplicación móvil para tomar nota
      de los pedidos de las mesas, impresora de comandas (para comunicación 
      con la cocina), gestión del \emph{stock} de productos en el almacén, 
      \dots
      \end{itemize} 
    \subsubsection{Gestión de pedidos a través de dispositivos móviles}
      \begin{itemize}
      \item \textbf{vMenu}
          % Los clientes pueden realizar pedidos desde su mesa a través de un
          %dispositivo móvil o fijo.
      \item \textbf{Brand Table}
          % la mesa de restaurante NFC que quiere eliminar al cajero de la
          %ecuación.
      \end{itemize}
    \subsubsection{Gestión de pagos}

% Local Variables:
%   coding: utf-8
%   mode: latex
%   mode: flyspell
%   ispell-local-dictionary: "castellano8"
% End:
