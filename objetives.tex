% -*- coding: utf-8 -*-

\chapter{Objetivos}
Para delimitar el alcance del proyecto se han definido una serie de
objetivos (generales y específicos) que se deben cumplir antes de dar
por finalizado el \acs{PFC}.

\section{Objetivo general}
Este PFC tiene como objetivo diseñar e implementar el prototipo de un sistema
que sea capaz de automatizar, complementar y enriquecer algunas de las tareas 
que se llevan a cabo en un restaurante, con objeto de aumentar su
productividad. Para conseguirlo, se hará uso del potencial que poseen los 
dispositivos móviles gracias a tecnologías inalámbricas como \acs{NFC} o
\emph{Bluetooth}. Esto permitirá obtener nuevas formas de interacción entre los 
clientes y el restaurante, pero sin poner en riesgo para ello la sencillez o la 
naturalidad del proceso.

El sistema a desarrollar no busca sustituir a los sistemas existentes en la
actualidad, sino que trata de complementarlos.

\section{Objetivos específicos}
Para la realización de este \acs{PFC} se han marcado los siguientes objetivos
específicos:
\subsubsection{Análisis}
\begin{enumerate}
\item Realizar un estudio de campo sobre el funcionamiento actual de los
  restaurantes. Cómo se organizan, cómo operan y qué medios utilizan para la
  realización de tareas.
\item Realizar una revisión sistemática de las aplicaciones existentes en el
  mercado que compartan las mismas premisas que nuestro sistema, para ver qué
  ofrecen y para ver cuáles son las ventajas y desventajas que implica elegir
  entre uno u otro sistema.
\item Investigar los estudios realizados de la introducción de la tecnología
  \acs{NFC} como modelo de negocio.
\item Identificar los requisitos que deben cumplir cada uno de los elementos
  que forman parte de nuestro sistema. Así como también los requisitos que
  deben satisfacer todos ellos funcionando como un conjunto.
\item Estudiar la forma de integrar este sistema con algún sistema que el
  restaurante ya tuviese implantado.
\item Estudiar la forma de introducir este sistema para clientes que no
  dispongan de un dispositivo adecuado.
\item Estudiar una forma de incentivar el uso del sistema de pago mediante
  el dispositivo móvil (descuentos, programas de fidelización, etc.).
\end{enumerate}

\subsubsection{Adquisición de conocimientos}
\begin{enumerate}
\item Adquirir los conocimientos necesarios para implementar interfaces de
  usuario utilizando la tecnología \acs{WPF} (\emph{Windows Presentation
  Foundation}).
\item Conocer las peculiaridades del lenguaje \acs{JavaME} (\emph{Java Micro
  Edition}) para la implementación de \emph{MIDLets} para dispositivos móviles
  con soporte \texttt{Java}.
\item Conocer las principales clases y métodos que ofrece la librería
  \emph{\acs{JSR}-257} para operar con etiquetas \acs{RFID} y dispositivos
  \acs{NFC} del entorno.
\item Conocer y usar las librerías implementadas como parte del estándar para
  \acs{JavaME} que permiten las comunicaciones vía \emph{Bluetooth} entre 
  dispositivos (\emph{\acs{JSR}-82}). Así como también, las librerías
  disponibles para \texttt{C\#}, como \texttt{InTheHand}.
\end{enumerate}

\subsubsection{Diseño e Implementación}
\begin{enumerate}
\item Diseñar e implementar una aplicación de escritorio táctil que permita
  configurar y actualizar el estado de las mesas del restaurante.
\item Implementar la forma de realizar el pago mediante el dispositivo móvil,
  utilizando para ello una interacción vía \emph{NFC}.
\item Diseñar e implementar una aplicación de escritorio táctil que permita
  registrar los pedidos de los clientes (de forma manual o automática - vía
  \emph{Bluetooth}), actualizar el estado de las mesas (con la información de
  los pedidos) y generar el importe total de cada cliente.
\item Diseñar e implementar una suite de \emph{MIDLets} que genere una lista
  de productos en el móvil a partir del contacto del dispositivo con las
  etiquetas \emph{RFID} que forman parte del menú. La aplicación también
  permitirá enviar al equipo de la barra el pedido vía \emph{Bluetooth}.
\item Diseñar la estructura de datos necesaria para representar la
  información de los platos del menú y de otras opciones (pedir la cuenta, 
  mostrar recomendaciones, etc.) dentro de las etiquetas \emph{RFID}.
\item Desarrollar un módulo en el propio sistema para la creación de
  recomendaciones atendiendo a las preferencias del cliente y a visitas
  anteriores al restaurante.
\end{enumerate}

\subsubsection{Pruebas}
\begin{enumerate}
\item Diseñar un plan de pruebas de campo, para comprobar la usabilidad y
la aceptación del uso de la aplicación móvil entre los potenciales
usuarios de un restaurante en el que está implantado el sistema
\texttt{MobiCarta}.
\item Diseñar un plan de pruebas de campo, para comprobar la usabilidad y
la aceptación del uso de las distintas aplicaciones de escritorio del
restaurante.
\end{enumerate}



% Local Variables:
%   coding: utf-8
%   fill-column: 90
%   mode: flyspell
%   ispell-local-dictionary: "american"
%   mode: latex
%   TeX-master: "main"
% End:
