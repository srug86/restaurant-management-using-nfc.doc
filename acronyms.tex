% -*- coding: utf-8 -*-

\chapter{Listado de acrónimos}

{\small
\begin{acronym}[XXXXXXXX]
  \Acro{CLDC}    {Connected Limited Device Configuration}

  \Acro{IDE}     {Electrically Erasable Programmable Read-Only Memory}

  \Acro{IDE}     {Integrated Development Environment}
  \Acro{ISO}     {International Organization for Standardization}

  \Acro{JavaME}  {Java Micro Edition}

  \Acro{MAmI}    {Modelling Ambient Intelligence}
  \Acro{MIDP}    {Mobile Information Device Profile}

  \acro{NDEF}    {NFC Data Exchange Format}
  \acro{NFC}     {Near Field Communication}

  \acro{OO}      {Orientación a Objetos}

  \Acro{PFC}      {Proyecto Fin de Carrera}

  \Acro{RFID}    {Radio Frequency IDentification}

  \Acro{SDK}     {Software Development Kit}

  \Acro{UCLM}    {Universidad de Castilla-La Mancha}
  \Acro{UML}     {Unified Modeling Language}
  \Acro{USB}     {Universal Serial Bus}

  \Acro{WiFi}    {Wireless Fidelity}
  \acro{WPF}     {Windows Presentation Foundation}
\end{acronym}
}


% \ac{OO}   la primera vez \acf, después \acs
% \acs{OO}  short: OO
% \acf{OO}  full : Object Oriented (OO)
% \acl{OO}  large: Object Oriented
% \acx{OO}         OO (Object Oriented)

% usa \Acro cuando no debe aparecer nunca expandido en el texto

% Local variables:
%   TeX-master: "main.tex"
% End:
