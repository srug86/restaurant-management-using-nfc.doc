% -*- coding: utf-8 -*-

\chapter{Listado de acrónimos}

{\small
\begin{acronym}[XXXXXXXX]
  \Acro{API}     {Application Programming Interface}
  \Acro{ARCO}    {Arquitectura y Redes de COmputadores}

  \Acro{CLDC}    {Connected Limited Device Configuration}

  \Acro{EEPROM}  {Electrically Erasable Programmable Read-Only Memory}
  \Acro{ESI}     {Escuela Superior de Informática}

  \Acro{GNU}     {GNU is Not Unix}
  \Acro{GPL}     {General Public License}
  \acro{GTA}     {Generic Terminal Application}

  \Acro{IDE}     {Integrated Development Environment}
  \Acro{IrDA}    {Infrared Data Association}
  \Acro{ISO}     {International Organization for Standardization}

  \Acro{JavaME}  {Java Micro Edition}

  \Acro{LPPL}    {\LaTeX Project Public License}

  \Acro{MAmI}    {Modelling Ambient Intelligence}
  \Acro{MIDP}    {Mobile Information Device Profile}

  \acro{NDEF}    {NFC Data Exchange Format}
  \acro{NFC}     {Near Field Communication}
  \acro{NMC}     {NFC-based Mobile Commerce}

  \acro{OO}      {Orientación a Objetos}

  \Acro{PFC}     {Proyecto Fin de Carrera}
  \Acro{PNG}     {Portable Network Graphics}
  
  \Acro{QR}      {Quick Response (Code)}

  \Acro{RFID}    {Radio Frequency IDentification}

  \Acro{SCD}     {Secure Common Domain}
  \Acro{SD}      {Secure Digital}
  \Acro{SDK}     {Software Development Kit}
  \Acro{SGBD}    {Sistema Gestor de Bases de Datos}
  \Acro{SIM}     {Subscriber Identity Module}
  \Acro{SMS}     {Short Message Service}
  \Acro{SQL}     {Structured Query Language}
  \Acro{SVC}     {System Version Control}
  \Acro{SVG}     {Scalable Vector Graphics}

  \Acro{TFG}     {Trabajo Fin de Grado}

  \Acro{UCLM}    {Universidad de Castilla-La Mancha}
  \Acro{UML}     {Unified Modeling Language}
  \Acro{USB}     {Universal Serial Bus}
  \Acro{UWB}     {Ultra-Wide Band}

  \Acro{WiFi}    {Wireless Fidelity}
  \acro{WPF}     {Windows Presentation Foundation}
  \Acro{W3C}     {World Wide Web Consortium}

  \acro{XML}     {eXtensible Markup Language}
\end{acronym}
}


% \ac{OO}   la primera vez \acf, después \acs
% \acs{OO}  short: OO
% \acf{OO}  full : Object Oriented (OO)
% \acl{OO}  large: Object Oriented
% \acx{OO}         OO (Object Oriented)

% usa \Acro cuando no debe aparecer nunca expandido en el texto

% Local variables:
%   TeX-master: "main.tex"
% End:
