% -*- coding: utf-8 -*-

\chapter{Introducción}
% Resumen de lo que va el PFC.

\section{Motivación}
El desarrollo de este \acs{PFC} intenta demostrar cómo pueden aplicarse las
distintas tecnologías que poseen los dispositivos móviles actuales, para
desarrollar nuevas formas de interacción, que permitan mejorar procesos
que se realizan habitualmente (como el caso de realizar pedidos o pagos en un
restaurante) y además, que lo hagan de una forma sencilla, cómoda e intuitiva.

Hoy en día, gran parte de las empresas de restauración cuentan con aplicaciones
que les ayudan a automatizar la anotación de pedidos y la generación de
facturas (e incluso conocer el estado de las mesas \cite{bib:saleYa}). Pero
estos sistemas simplemente acomodan el uso del clásico lápiz y papel y la
máquina registradora. Este sistema va un paso más allá y trata de mejorar 
visiblemente la productividad del proceso, así como un enriquecimiento del 
entorno, además de proponer una forma innovadora de realizar pedidos.

Actualmente, hay algunas implementaciones que buscan automatizar la forma
de realizar pedidos, como \emph{vMenu} (de la empresa \emph{vloo}), que
utiliza un \emph{iPad}, tablet o pantalla, como carta para realizar pedidos 
\cite{bib:vMenu}. La solución que se propone en el presente \acs{PFC}, parte de
la misma idea pero teniendo en cuenta una mínima interacción del usuario con el 
dispositivo para la obtención de servicios (p.e. petición de pedidos), una 
mayor versatilidad (incluyendo la realización de pagos y generación de 
recomendaciones), y un menor coste (pues no hace falta disponer de tantos 
dispositivos como mesas haya).

\section{Organización del documento}
% Resumen de cómo está estructurado el documento y qué se puede encontrar
% en cada capítulo.
El presente documento está dividido en seis capítulos. A continuación, se
describe brevemente cuál es el contenido de cada uno ellos:
\begin{itemize}
\item \textbf{Capítulo 1: <<Introducción>>}

Es el presente capítulo. Expone de forma general cuál es la motivación del
desarrollo de este \acs{PFC}.

\item \textbf{Capítulo 2: <<Estado del arte>>}

Presenta las principales tecnologías que van a servir de base teórica para el 
proyecto y cita algunos de los estudios realizados y aplicaciones 
comerciales existentes que están relacionadas con el tema.

\item \textbf{Capítulo 3: <<Objetivos>>}

Enumera los objetivos generales y específicos que se pretenden cumplir con la  
elaboración de este \acs{PFC}.

\item \textbf{Capítulo 4: <<Métodos de trabajo y herramientas>>}

Define cuál va a ser la metodología de trabajo a seguir para
alcanzar los objetivos marcados y cuáles van a ser las herramientas
\emph{hardware} y \emph{software} sobre las que se va a apoyar para
conseguirlo.

\item \textbf{Capítulo 5: <<Resultados>>}

Detalla cada una de las fases que se han derivado de la aplicación de la
metodología y de las herramientas escogidas en el capítulo anterior.

\item \textbf{Capítulo 6: <<Conclusiones y propuestas>>}

Resume cuáles han sido finalmente los objetivos que se han alcanzado y cuáles
pueden ser las nuevas líneas de trabajo futuro.

\item \textbf{Anexos}

Contiene todos aquellos documentos de interés, que se han ido generando durante 
la elaboración del \acs{PFC}.

\end{itemize}

% Local Variables:
%   coding: utf-8
%   mode: latex
%   mode: flyspell
%   ispell-local-dictionary: "castellano8"
% End:
