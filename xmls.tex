\chapter{Definición de los elementos \acs{XML} utilizados}
\label{chap:xmls}
La comunicación entre los elementos del sistema es una parte fundamental del
mismo. Como los elementos están implementados utilizando varios lenguajes y
tecnologías, sólamente se permite la comunicación a través de datos primitivos
(enteros, decimales, cadenas, etc.). Pero debido a la complejidad de los datos
que se desean compartir, será necesario estructurarlos en formato \acs{XML}. 
Esto permitirá los intercambios de cadenas que contengan información
estructurada.

A continuación, se citan todas las estructuras definidas para compartir
información entre elementos del sistema.

\section{Elementos generados por los \emph{servicios web}}
Son los elementos definidos para realizar comunicaciones entre los
\emph{servicios web} y las aplicaciones de escritorio del restaurante
(\emph{recibidor} y \emph{barra}).
% Elementos generados por los servicios web
\subsection{\acs{XML} para listar los escenarios disponibles}
El elemento \texttt{rooms} contiene un listado de todos los escenarios 
definidos con anterioridad en el editor de la aplicación del \emph{recibidor}. 
Los destinatarios de este tipo de elementos son las dos aplicaciones de 
escritorio del restaurante (\emph{recibidor} y \emph{barra}). El listado
\ref{code:DTDrooms} muestra el esquema seguido por los elementos de este tipo:
% rooms.dtd
\lstinputlisting[
language=XML,
caption={\acs{DTD} que describe la estructura de los documentos
  \emph{rooms.xml}.},
label=code:DTDrooms]
{listings/rooms.dtd}
El listado \ref{code:XMLrooms} muestra un ejemplo de un elemento de tipo
\texttt{rooms}. En este caso, la base de datos tiene almacenados dos
escenarios distintos: \texttt{Room1} y \texttt{Room2}.
% rooms.xml
\lstinputlisting[
language=XML,
caption={Ejemplo \acs{XML} que describe el resumen de dos habitaciones.},
label=code:XMLrooms]
{listings/rooms.xml}

\subsection{\acs{XML} que describe el estado de las mesas del restaurante}
El elemento \texttt{tables} contiene una lista actualizada del estado de las
mesas del restaurante. Esta información es consumida por las aplicaciones
de escritorio del restaurante (\emph{recibidor} y \emph{barra}). El listado
\ref{code:DTDtables} muestra el esquema que siguen estos elementos:
% tables.dtd
\lstinputlisting[
language=XML,
caption={\acs{DTD} que describe la estructura de los documentos
  \emph{tables.xml}.},
label=code:DTDtables]
{listings/tables.dtd}
El listado \ref{code:XMLtables} muestra un ejemplo de la situación de las
mesas del restaurante en un momento determinado:
% tables.xml
\lstinputlisting[
language=XML,
caption={Ejemplo \acs{XML} que describe el estado de las nueve mesas del
  restaurante.},
label=code:XMLtables]
{listings/tables.xml}

\subsection{\acs{XML} para listar los pedidos recibidos durante la jornada}
El elemento \texttt{orders} describe una lista con los pedidos registrados
por el sistema dentro de una misma jornada. La destinataria es la aplicación
de la \emph{barra}, que es la que gestiona todos los pedidos. El listado
\ref{code:DTDorders} muestra la definición \acs{DTD} para este tipo de
elementos:
% orders.dtd
\lstinputlisting[
language=XML,
caption={\acs{DTD} que describe la estructura de los documentos
  \emph{orders.xml}.},
label=code:DTDorders]
{listings/orders.dtd}
El listado \ref{code:XMLorders} muestra el \acs{XML} de un elemento de tipo
\texttt{orders} en el que aparecen listados dos pedidos:
% orders.xml
\lstinputlisting[
language=XML,
caption={Ejemplo \acs{XML} que describe los pedidos recibidos durante la
  jornada.},
label=code:XMLorders]
{listings/orders.xml}

\subsection{\acs{XML} que describe el estado de una mesa}
% table.dtd
\lstinputlisting[
language=XML,
caption={\acs{DTD} que describe la estructura de los documentos
  \emph{table.xml}.},
label=code:DTDtable]
{listings/table.dtd}
% table.xml
\lstinputlisting[
language=XML,
caption={Ejemplo \acs{XML} que describe cuál es el estado de una mesa, quién 
  la ocupa y qué pedidos tiene asociados.},
label=code:XMLtable]
{listings/table.xml}

\subsection{\acs{XML} para listar el histórico de pedidos del restaurante}
El elemento \texttt{historical} recoge un listado con el histórico de todos los 
pedidos que han sido facturados por el sistema. Este histórico tiene como
destinatario la aplicación de la \emph{barra}. El listado
\ref{code:DTDhistorical} muestra la definición \acs{DTD} de este tipo de 
elementos:
% historical.dtd
\lstinputlisting[
language=XML,
caption={\acs{DTD} que describe la estructura de los documentos
  \emph{historical.xml}.},
label=code:DTDhistorical]
{listings/historical.dtd}
El listado \ref{code:XMLhistorical} muestra un ejemplo de un listado de tipo
\texttt{historical}:
% historical.xml
\lstinputlisting[
language=XML,
caption={Ejemplo \acs{XML} que describe el histórico de pedidos del 
  restaurante.},
label=code:XMLhistorical]
{listings/historical.xml}

\subsection{\acs{XML} que describe la factura de una mesa}
El elemento de tipo \texttt{bill} describe los datos necesarios para
facturar a un cliente. Para ello se recoge información de distintas tablas
de la base de datos y se calculan los campos restantes. La aplicación de la
\emph{barra} es la destinataria de este tipo de elementos. La figura 
\ref{code:DTDbill} muestra la definición de este tipo de dato:
% bill.dtd
\lstinputlisting[
language=XML,
caption={\acs{DTD} que describe la estructura de los documentos
  \emph{bill.xml}.},
label=code:DTDbill]
{listings/bill.dtd}
En el listado \ref{code:XMLbill} puede verse el ejemplo de un elemento de tipo
\texttt{bill}. Este elemento contiene datos acerca del establecimiento, del
cliente, de la propia factura, de los pedidos registrados y del importe a
abonar:
% bill.xml
\lstinputlisting[
language=XML,
caption={Ejemplo \acs{XML} que describe la factura detallada para la mesa 2.},
label=code:XMLbill]
{listings/bill.xml}

\subsection{\acs{XML} que describe la factura reducida de una mesa}
% shortBill.dtd
\lstinputlisting[
language=XML,
caption={\acs{DTD} que describe la estructura de los documentos
  \emph{shortBill.xml}.},
label=code:DTDshortBill]
{listings/shortBill.dtd}
% shortBill.dtd
\lstinputlisting[
language=XML,
caption={Ejemplo \acs{XML} que describe la versión reducida de la factura para
  la mesa 2.},
label=code:XMLshortBill]
{listings/shortBill.xml}

\subsection{\acs{XML} que lista las recomendaciones para un cliente \acs{NFC}}
%%%%%%%%%%%%%%%%%%%%%%%%%%%%%%%%%%%%%%%%%
%%%%%%%%%%%%%%%%%%%%%%%%%%%%%%%%%%%%%%%%%
% recommendation.dtd
\lstinputlisting[
language=XML,
caption={\acs{DTD} que describe la estructura de los documentos
  \emph{recommendation.xml}.},
label=code:DTDrecommendation]
{listings/recommendation.dtd}
% recommendation.xml
\lstinputlisting[
language=XML,
caption={Ejemplo \acs{XML} que describe las recomendaciones calculadas para
  uno de los clientes \acs{NFC} del restaurante \emph{mobiCarta}.},
label=code:XMLrecommendation]
{listings/recommendation.xml}

\section{Elementos generados por la \emph{aplicación móvil}}
Son los elementos definidos para la comunicación entre la \emph{aplicación
móvil} y las aplicaciones de escritorio del restaurante (\emph{recibidor} y
\emph{barra}).
% Elementos generados por la aplicación móvil
\subsection{\acs{XML} que recoge los datos personales de un cliente \acs{NFC}}
El elemento \texttt{profile} recoge los datos personales introducidos por el
cliente a través de la aplicación móvil. Estos datos son guardados, en primer
lugar, en el propio dispositivo móvil (como \texttt{profile.xml}), para que la
aplicación móvil pueda disponer de ellos cuando lo necesite. Durante el
proceso de registro en el restaurante, esta información viajará vía
\emph{Bluetooth} a la aplicación del \emph{recibidor}. El listado
\ref{code:DTDprofile} muestra la definición \acs{DTD} para este tipo de
elementos:
% profile.dtd
\lstinputlisting[
language=XML,
caption={\acs{DTD} que describe la estructura de los documentos
  \emph{profile.xml}.},
label=code:DTDprofile]
{listings/profile.dtd}
El listado \ref{code:XMLprofile} muestra un ejemplo de los datos almacenados
por un elemento de tipo \texttt{profile}:
% profile.xml
\lstinputlisting[
language=XML,
caption={Ejemplo \acs{XML} que describe los datos personales de un cliente
  \acs{NFC}.},
label=code:XMLprofile]
{listings/profile.xml}

\subsection{\acs{XML} que describe los elementos de un pedido \acs{NFC}}
Un elemento de tipo \texttt{clientOrder} define la lista de productos de
un pedido, elaborada por la aplicación móvil mediante la tecnología \acs{NFC}.
Esta lista se transmite vía \emph{Bluetooth} a la aplicación de la
\emph{barra}. El listado \ref{code:DTDclientOrder} describe la definición de
los elementos de este tipo:
% clientOrder.dtd
\lstinputlisting[
language=XML,
caption={\acs{DTD} que describe la estructura de los documentos
  \emph{clientOrder.xml}.},
label=code:DTDclientOrder]
{listings/clientOrder.dtd}
El listado \ref{code:XMLclientOrder} muestra un ejemplo en el que el cliente
\texttt{05692939w} solicita dos unidades de dos productos distintos:
% clientOrder.xml
\lstinputlisting[
language=XML,
caption={Ejemplo \acs{XML} que describe los pedidos realizados por un
  cliente \acs{NFC}.},
label=code:XMLclientOrder]
{listings/clientOrder.xml}

\section{Elementos generados por la \emph{aplicación del recibidor}}
Son los elementos definidos para comunicar la aplicación del \emph{recibidor}
con los \emph{servicios móviles}.
% Elementos generados por la aplicación del recibidor
\subsection{\acs{XML} que describe los datos del restaurante}
% restaurant.dtd
\lstinputlisting[
language=XML,
caption={\acs{DTD} que describe la estructura de los documentos
  \emph{restaurant.xml}.},
label=code:DTDrestaurant]
{listings/restaurant.dtd}
% restaurant.xml
\lstinputlisting[
language=XML,
caption={Ejemplo \acs{XML} que describe los datos de interés del restaurante.},
label=code:XMLrestaurant]
{listings/restaurant.xml}

\subsection{\acs{XML} que describe los elementos físicos de un escenario}
% room.dtd
\lstinputlisting[
language=XML,
caption={\acs{DTD} que describe la estructura de los documentos
  \emph{room.xml}.},
label=code:DTDroom]
{listings/room.dtd}
% room.xml
\lstinputlisting[
language=XML,
caption={Ejemplo \acs{XML} que describe los elementos físicos que forman
  parte del escenario de nombre \emph{Room1}.},
label=code:XMLroom]
{listings/room.xml}

\section{Elementos generados por la \emph{aplicación de la barra}}
La aplicación de la \emph{barra} sólamente genera un tipo de elemento
\acs{XML} para transmitir información a los \emph{servicios web}:
% Elementos generados por la aplicación de la barra
\subsection{\acs{XML} que lista los productos del restaurante}
Un elemento de tipo \texttt{products} recoge la lista completa de todos los
productos que ofrece (o que ha ofrecido) el restaurante. Los elementos de
esta lista están almacenados en una misma tabla de la base de datos y sólo
puede ser modificada a través de la aplicación de la \emph{barra}. El
listado \ref{code:DTDproducts} muestra la definición de este tipo de elementos:
% products.dtd
\lstinputlisting[
language=XML,
caption={\acs{DTD} que describe la estructura de los documentos
  \emph{products.xml}.},
label=code:DTDproducts]
{listings/products.dtd}
El listado \ref{code:XMLproducts} muestra un ejemplo del listado de productos
de un restaurante:
% products.xml
\lstinputlisting[
language=XML,
caption={Ejemplo \acs{XML} que describe la carta de productos que maneja el
  restaurante.},
label=code:XMLproducts]
{listings/products.xml}

