\chapter{La \acs{API} \texttt{\acs{JSR}-82} (\texttt{Bluetooth})}
\label{chap:jsr82}

A continuación, se va a hacer un repaso de las principales clases y métodos
que forman parte del paquete \texttt{javax.bluetooth}, que está orientado a 
las operaciones \texttt{Bluetooth} que no utilizan el protocolo \acs{OBEX}.
Como, en este caso, la aplicación móvil siempre va a ejercer como 
cliente, en el esquema \emph{cliente-servidor}, sólo van a exponerse las
clases orientadas a las comunicaciones cliente, en este caso las de tipo
\acs{SPP}\footnote{El \emph{Serial Port Profile} es un perfil \texttt{Bluetooth} 
basado en la \texttt{\acs{ETSI} 07,10} y en el protocolo \acs{RFCOMM}, que
emula un cable serial para proporcionar un sustituto similar a la interfaz
\texttt{RS-232}, incluyendo sus principales señales de
control.}~\cite{bib:jsr82}:
\begin{itemize}
\item En primer lugar, antes de realizar cualquier otra operación, es necesario
iniciar la interfaz \texttt{Bluetooth} del dispositivo. Para ello:
  \begin{itemize}
  \item primero se crea una instancia de tipo \texttt{LocalDevice}, que va a 
  hacer referencia al dispositivo local. La clase \texttt{LocalDevice} es un 
  ``singleton'', así que para obtener su única instancia existente se llamará 
  al método \texttt{getLocalDevice()}.

  \begin{lstlisting}[
    language = Java]
    LocalDevice localDevice = null;
    try {
      localDevice = LocalDevice.getLocalDevice();
    } catch (BluetoothStateException e) { }
  \end{lstlisting}

  Este objeto será el punto de partida de prácticamente cualquier operación
  que se vaya a llevar a cabo con esta \acs{API}.

  \item Y después, mediante el método \texttt{setDiscoverable/1}, se activa la 
  conectividad del dispositivo. Los posibles valores que admite este método
  están definidos en la clase \texttt{DiscoveryAgent} como campos estáticos y
  son los siguientes:
    \begin{itemize}
    \item \texttt{DiscoveryAgent.GIAC} (\emph{General/Unlimited Inquiry Access
    Code}): permite una conectividad ilimitada.
    \item \texttt{DiscoveryAgent.LIAC} (\emph{Limited Dedicated Inquiry Access
    Code}): permite una conectividad limitada.
    \item \texttt{DiscoveryAgent.NOT\_DISCOVERABLE}: hace que el dispositivo
    no sea visible para el resto de dispositivos.
    \end{itemize}
  Por ejemplo, para permitir una conectividad ilimitada:

  \begin{lstlisting}[
    language = Java]
    localDevice.setDiscoverable(DiscoveryAgent.GIAC);
  \end{lstlisting}
  \end{itemize}

\item El siguiente paso será la búsqueda de dispositivos y servicios. Las
búsquedas se realizarán a través del objeto \texttt{DiscoveryAgent}. Este
objeto es único y se obtendrá a través del método \texttt{getDiscoveryAgent()}
del objeto \texttt{LocalDevice}:

\begin{lstlisting}[
  language = Java]
  DiscoveryAgent discoveryAgent =
    LocalDevice.getLocalDevice().getDiscoveryAgent();
\end{lstlisting}
  \begin{itemize}
  \item Para comenzar una nueva \textbf{búsqueda de dispositivos}, se hará uso 
  del método \texttt{startInquiry()}. Este método requiere dos argumentos. El 
  primero especifica el modo de conectividad que deben tener los dispositivos a 
  buscar (\texttt{DiscoveryAgent.GIAC} o \texttt{DiscoveryAgent.LIAC}).
  El segundo es un objeto que implementa la interfaz
  \texttt{DiscoveryListener}. A través de este último, serán notificados los 
  dispositivos que se vayan descubriendo:

  \begin{lstlisting}[
    language = Java]
    discoveryAgent.startInquiry(DiscoveryAgent.GIAC, this);
  \end{lstlisting}

  Para cancelar la búsqueda se hará uso del método \texttt{cancelInquiry()}.

  \begin{lstlisting}[
    language = Java]
    discoveryAgent.cancelInquiry(this);
  \end{lstlisting}

  La interfaz \texttt{DiscoveryListener} contiene dos métodos para la 
  búsqueda de dispositivos:
    \begin{itemize}
    \item \texttt{public void \textbf{deviceDiscovered}(RemoteDevice rd,
      DeviceClass c)}

    Cada vez que se descubre un dispositivo se llama a este método. El primer
    argumento representa el dispositivo encontrado. El segundo permite 
    determinar el tipo de dispositivo.

    \item \texttt{public void \textbf{inquiryCompleted}(int c)}

    Este método es llamado cuando la búsqueda de dispositivos ha finalizado. El
    argumento entero indica el motivo de la finalización.
    \end{itemize}

  \item Para comenzar una \textbf{búsqueda de servicios}, se utilizará el
  método \texttt{searchServices()}, de la clase \texttt{DiscoveryAgent}. Este
  método tiene cuatro atributos:
    \begin{itemize}
    \item El primero especifica los atributos de servicio que van a ser objeto
    de búsqueda\footnote{Los servicios son descritos a través de unos atributos 
    que son identificados numéricamente. Los identificadores de estos atributos 
    están contenidos por un \emph{array} de enteros.}.
    \item El segundo es un \emph{array} de identificadores de servicio y
    permiten especificar los servicios que van a ser objeto de la búsqueda.
    \item El tercero representa al dispositivo sobre el que se va a realizar la
    búsqueda.
    \item Y el último es un objeto que implementa la interfaz
    \texttt{DiscoveryListener} y será usado para notificar los eventos de 
    búsqueda de servicio.
    \end{itemize}
  Además, devuelve un valor entero que identifica la búsqueda. Este valor
  entero se utiliza para determinar a qué búsqueda pertenecen los eventos
  \texttt{servicesDiscovered()} y \texttt{serviceSearchCompleted()}, que son
  los otros dos métodos que define la interfaz \texttt{DiscoveryListener} (en
  este caso para la búsqueda de servicios).
    \begin{itemize}
    \item \texttt{public void \textbf{servicesDiscovered}(int transID,
      ServiceRecord[] sr)}

    Cada vez que se descubre un servicio se llama a este método. El primer
    argumento identifica el proceso de búsqueda. Es el mismo que devolvió
    \texttt{searchDevices()} cuando se comenzó la búsqueda. El segundo, es un
    \emph{array} de objetos \texttt{ServiceRecord}, que describe las
    características del servicio \texttt{Bluetooth}.

    \item \texttt{public void \textbf{serviceSearchCompleted}(int transID, int 
    respCode)}

    Este método es llamado cuando la búsqueda de servicios ha finalizado. El
    primer argumento identifica el proceso de búsqueda (valor devuelto por al
    invocar el método \texttt{searchServices()}. El segundo argumento indica el
    motivo de finalización de la búsqueda.
    \end{itemize}

  Para cancelar un proceso de búsqueda de servicios se hará uso del método
  \texttt{cancelServiceSearch()}, pasándole como argumento el identificador
  del proceso de búsqueda.
  \end{itemize}

\item Por último, para establecer una comunicación:
  \begin{itemize}
  \item en primer lugar, se obtendrá la \acs{URL} necesaria para establecer
  la conexión a través del método \texttt{getConnectionURL()} de un objeto
  \texttt{ServiceRecord} (obtenido previamente a través del método
  \texttt{servicesDiscovered}). Este método requiere dos argumentos: el
  primero, indica si se debe autentificar y/o cifrar la conexión; y el
  segundo, especifica si el dispositivo ejercerá como maestro (\texttt{true})
  o bien, no importa si es maestro o esclavo (\texttt{false}):

  \begin{lstlisting}[
    language = Java]
    ServiceRecord sr = ...;
    String url =
      sr.getConnectionURL(ServiceRecord.NOAUTHENTICATE_NOENCRYPT, false);
  \end{lstlisting}

  \item Una vez obtenida la \acs{URL}, se usará el método
  \texttt{Connector.open()} para realizar la conexión. Este método devolverá un 
  objeto distinto según el tipo de protocolo usado. En este caso (cliente
  \acs{SPP}), devolverá un \texttt{StreamConnection}. A partir del
  \texttt{StreamConnection} podremos obtener los flujos de entrada y salida:

  \begin{lstlisting}[
    language = Java]
    StreamConnection conn =
      (StreamConnection) Connector.open(url);
    OutputStream out = con.openOutputStream();
    InputStream in = con.openInputStream();
  \end{lstlisting}
  \end{itemize}

\item Para dar por concluida la comunicación, no conviene olvidar que hay que 
cerrar los flujos de entrada y salida y la propia conexión:

  \begin{lstlisting}[
    language = Java]
    finally {
      try {
        if (in != null)
          in.close();
        if (out != null)
          out.close();
        if (conn != null)
          conn.close();
      } catch (IOException e) { }
    }
  \end{lstlisting}
\end{itemize}


