\chapter{Definición de las etiquetas \acs{RFID} utilizadas}
\label{chap:tags}
Uno de los elementos más importantes del sistema de \emph{mobiCarta} son las etiquetas \acs{RFID}. Todas las etiquetas utilizadas son de la clase
\emph{Mifare} y tienen una capacidad de 1KB. Dependiendo del tamaño del
restaurante, el sistema puede contener más o menos etiquetas, pero todas ellas
se encuentran en la \emph{zona del recibidor} o en los menús de las mesas y 
pertenecen a uno de estos cinco tipos:
\begin{description}
\item[Etiqueta del recibidor]:
  \begin{itemize}
  \item \textbf{app/checkpoint} .- Es uno de los tipos contenidos en el
  \emph{Push Registry} del dispositivo móvil. Por lo tanto, cuando un
  dispositivo móvil contacta con una etiqueta de este tipo, inicia 
  automáticamente la aplicación, esta vez en \emph{modo registro}. Todos los 
  \emph{clientes \acs{NFC}} deben tocar esta etiqueta para registrar su paso
  por la \emph{zona del recibidor} para confirmar su llegada o su salida del 
  restaurante o su intención de pagar vía \acs{NFC}. Este tipo de etiqueta no 
  contiene ninguna información adicional.
  \end{itemize}

\item[Etiquetas de los menús]:
  \begin{itemize}
  \item \textbf{app/product} .- Es otro de los tipos contenidos en el
  \emph{Push Registry} del dispositivo móvil. El contacto con esta etiqueta
  inicia la aplicación en \emph{Modo pedido}. Cada una de las etiquetas de
  este tipo contiene la información del nombre del producto al que representa.
  Por lo tanto, cada contacto con una de estas etiquetas significa añadir
  una unidad de ese tipo de producto a la lista del pedido (incluyendo el
  primer contacto que es el que inicia la aplicación).
  \item \textbf{app/subtract-product} .- Este tipo de etiquetas sólo son
  captadas por la aplicación móvil cuando esta se encuentra en \emph{Modo
  pedido}. Sirve para decrementar el número de unidades de un tipo de
  producto de la lista del pedido: primero, tocando con el móvil esta etiqueta;
  y después, tocando la etiqueta del producto que se desea decrementar.
  Cada carta o menú dispondrá al menos de una etiqueta de este tipo, en un
  lugar visible. Este tipo de etiqueta no contiene ninguna otra información 
  adicional.
  \item \textbf{app/send-order} .- Al igual que la anterior, este tipo de
  etiquetas son captadas por la aplicación cuando está en \emph{Modo pedido}.
  Sirve para enviar la lista del pedido confeccionada a la aplicación de la
  \emph{barra}. Cada carta dispondrá de una etiqueta de este tipo.
  Esta etiqueta no contiene ningún otro tipo de información.
  \item \textbf{app/bill-request} .- Es el tercer tipo de etiquetas que el
  \emph{Push Registry} contiene para esta aplicación móvil. El contacto con
  una etiqueta de este tipo arranca la aplicación en \emph{Modo solicitar
  factura}. Este modo genera una llamada vía \emph{Bluetooth} a la aplicación
  de la \emph{barra}, para que esta le devuelva un resumen con la factura de
  los pedidos que el cliente ha realizado. Cada carta dispondrá de una
  etiqueta de este tipo. Este tipo de etiqueta tampoco contiene ninguna
   información adicional.
  \end{itemize}
\end{description}

Para leer y escribir rápidamente el tipo y el contenido de todas las etiquetas
se han desarrollado dos mini-aplicaciones (\emph{Reader} y \emph{Writer},
respectivamente). Estas dos mini-aplicaciones pueden encontrarse dentro del
contenido extra del CD que acompaña a la presente memoria.

