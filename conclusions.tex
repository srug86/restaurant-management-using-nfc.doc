% -*- coding: utf-8 -*-

\chapter{Objetivos alcanzados y propuestas de mejora}
\drop{E}ste último capítulo resume cuáles han sido los objetivos alcanzados
y recoge algunas propuestas de cuáles pueden ser las nuevas líneas de trabajo 
futuro.

\section{Objetivos alcanzados}
En el capítulo \ref{chap:objectives}, se proponía como objetivo principal el
\emph{diseñar e implementar el prototipo de un sistema que fuera capaz de 
automatizar, complementar y enriquecer algunas de las tareas que se llevan a 
cabo en un restaurante}, utilizando para ello el potencial que ofrecen los
dispositivos móviles, a través de tecnologías como \acs{NFC} o
\texttt{Bluetooth}. Para alcanzar este objetivo final, se especificaron varios
objetivos específicos (sección \ref{sec:specificO}) a cumplir.

Después de observar los resultados de la evaluación del sistema desarrollado
(sección \ref{sec:testing}), puede concluirse que, los objetivos específicos y 
por extensión el objetivo principal, han sido satisfechos con relativo éxito.

El sistema desarrollado permite gestionar toda la información que se genera en
un restaurante de forma automática: ocupación de mesas, atención de pedidos, 
generación de facturas, etc. Además, complementa estas funcionalidades con la
posibilidad de atender de forma eficaz y personalizada a los clientes que
utilizan la aplicación móvil implementada.

De cara al cliente, el uso de la aplicación móvil le permite: realizar pedidos, 
conocer el importe total de lo consumido y acceder a un programa de
fidelización de clientes (todo ello a través de interacciones sencillas e
intuitivas con su dispositivo móvil).

El programa de fidelización ayuda a incentivar el uso de la
aplicación móvil y se fundamenta en la implementación de un módulo 
recomendador. Este módulo utiliza los datos del historial del cliente, generado 
por sus interacciones con el sistema, para ofrecerle una serie de descuentos
(que premien su fidelidad) y para recomendarle el consumo de productos que 
puedan interesarle.

De cara al restaurante, la generación y almacenamiento de datos como: el 
historial de pedidos o el listado de clientes, puede resultar de gran utilidad
a la hora de conocer qué productos tienen una mayor aceptación y cuáles son
los hábitos de los clientes habituales.

El único objetivo específico que no se ha completado es el de la 
\emph{implementación de una forma de pago mediante \acs{NFC}}. En este caso,
el sistema sólamente simula la interacción.

\section{Propuestas de mejora}
Aunque el sistema desarrollado cumple con los principales objetivos marcados,
no hay que olvidar que sólo es un \textbf{prototipo}, no un sistema final. Y 
por tanto, se han pasado por alto fases tan importantes como la realización de
pruebas unitarias, de integración, de sistema y funcionales; y esto significa
que el software puede no estar libre de errores y que la calidad de sus
elementos no está demostrada.

Por otro lado, conviene tener muy en cuenta las pruebas de campo realizadas
(sección \ref{sec:testing}), a la hora de mejorar la usabilidad y ampliar las
funcionalidades de las aplicaciones desarrolladas.

Otros aspectos a mejorar son los siguientes:
\begin{itemize}
\item \emph{\textbf{Seguridad}}. El proyecto desarrollado no ha tenido en 
cuenta la implementación de ningún mecanismo extra de seguridad, más allá de 
los mecanismos propios de cada una de las tecnologías (\emph{servicios web},
\emph{bases de datos}, \texttt{Bluetooth} o \acs{NFC}). Uno de los elementos
que no cuenta con ningún mecanismo de protección son las etiquetas
\acs{RFID}. Esta información es de carácter público (no es confidencial) pero, 
al no estar protegida, puede ser modificada ``a drede'' por cualquiera que 
tenga un programa \emph{escritor} de etiquetas. Otra de las prácticas que
aumentarían la seguridad sería la implementación de un \emph{login}, con el
que se controle el acceso a la información con la que trabajan las aplicaciones
que forman parte del restaurante.
\item \emph{\textbf{Comunicaciones}}. Las comunicaciones entre el cliente y el
restaurante se realizan mediante conexiones serie con \texttt{Bluetooth}.
Esta tecnología cuenta con una serie de desventajas y limitaciones (ver sección
\ref{subsec:disadvantages}) que podrían solucionarse utilizando la tecnología
\acs{WiFi}.
\item \emph{\textbf{Flexibilidad}}. Como ya se comentó en la sección
\ref{subsec:scenario}, el sistema está diseñado para atender a las necesidades
de los restaurantes de tipo \emph{gourmet} y de tipo \emph{temático} (que
son los que cuentan con la figura del \emph{maître}). Una posible mejora
consistiría en adaptar el sistema a restaurantes de tipo \emph{fast food} o
\emph{take away} (que no tienen control de entrada).
\item \emph{\textbf{Funcionalidades}}. Aparte de la función de la \emph{caja 
registradora} (propuesta realizada por los usuarios durante la etapa de
pruebas), las funcionalidades del sistema se pueden ampliar añadiendo, por
ejemplo:
\begin{itemize}
\item Un módulo que gestione pedidos \emph{para llevar}.
\item Un módulo que permita reservar mesas (manual o automáticamente).
\item Mejorar el visor de estadísticas, que permita, por ejemplo, hacer el 
balance de ingresos de la jornada.
\item Completar la información almacenada por las etiquetas de los productos 
de la carta. De esta forma el cliente podría consultar otros datos de interés 
como: ingredientes, preparación, precio, etc.
\item Añadir la opción de imprimir las facturas generadas.
\item Incluir una sección de manuales y ayuda.
\item Y por supuesto, implementar las condiciones necesarias para implantar
un sistema de cobro mediante \acs{NFC}: aprovechando alguno de los
servicios ya desarrollados, como \texttt{Google Wallet}, o instalando alguno
de los terminales de pago dedicados, como \texttt{Visa PayWave}.
\end{itemize}
\end{itemize}

En cualquier caso, el aspecto más importante para que el sistema desarrollado 
llegue algún día a implantarse en restaurantes reales, es la \textbf{promoción 
de la tecnología \acs{NFC}} que, si bien en lugares como Japón, Corea del Sur 
o Estados Unidos, ya goza de una gran popularidad; todavía no ha logrado 
establecerse con firmeza en el Europa. Y aquí es donde se plantéa la clásica 
\emph{pregunta del huevo y la gallina}: ¿debe invertirse dinero en sacar al 
mercado dispositivos que hagan uso de la tecnología \acs{NFC} en un mercado 
que no cuenta con aplicaciones que le saquen provecho? O ¿deben primero 
implementarse aplicaciones que estén preparadas para utilizar esta tecnología, 
aún sabiendo que todavía son pocos los dispositivos que van a poder aprovechar 
estos servicios? De momento, según lo visto en figura \ref{fig:nfcGraph} 
(página \pageref{fig:nfcGraph}), se está apostando por la primera alternativa. 
Por lo que, no está demás pensar, que la venta de esta clase de dispositivos 
incentivará la implementación de nuevos sistemas y aplicaciones que los
aprovechen.




% Local Variables:
%   coding: utf-8
%   mode: latex
%   mode: flyspell
%   ispell-local-dictionary: "castellano8"
% End:
