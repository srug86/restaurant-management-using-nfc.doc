\chapter{La \acs{API} \texttt{\acs{JSR}-257} (\emph{Contactless Communication})}
\label{chap:jsr257}

En el presente anexo, se hace un repaso de los conceptos básicos que
forman parte del trabajo con etiquetas inteligentes y su implementación a
partir de la \acs{API} \texttt{\acs{JSR}-257}~\cite{bib:jfontecha}:
\begin{itemize}
\item \textbf{Implementación de interfaces}. Para poder hacer uso de las 
funcionalidades \acs{NFC} por medio de \acs{JavaME}
es necesario implementar al menos la interfaz \texttt{TargetListener} en la
clase principal de la aplicación. Esto permitirá la lectura y escritura de
etiquetas una vez iniciada la aplicación.

Si además se desea que, ante el evento de contacto con una etiqueta, sea el 
propio móvil el que inicie la aplicación, se deberá implementar la interfaz
\texttt{NDEFRecordListener}, para provocar el ``autoarranque'' de la aplicación
ante un tipo de etiqueta definido.

Implementar tanto \texttt{TargetListener} como \texttt{NDEFRecordListener},
implica sobreescribir los métodos de dichas interfaces,
\texttt{recordDetected/1} y \texttt{targetDetected/1} respectivamente
(listado \ref{code:extendsNFC}):

\begin{lstlisting}[
  language = Java,
  caption = {Implementación de las interfaces \texttt{TargetListener} y
  \texttt{NDEFRecordListener}.},
  label = code:extendsNFC]
  public class MiMIDLet extends MIDlet implements NDEFRecordListener, TargetListener {
    @override
    void recordDetected(NDEFMessage ndefMessage) { ... }
    @override
    void targetDetected(TargetProperties[] properties) { ... }
  }
\end{lstlisting}

\item \textbf{Descubrimiento de dispositivos y etiquetas}. Para poder leer y 
escribir etiquetas \acs{RFID} con \acs{JavaME}, se necesita realizar un 
descubrimiento previo y añadir el \emph{Listener} correspondiente. De esta 
forma en el constructor del \emph{MIDlet} o nada más iniciar su ejecución, 
podrá realizarse su registro.

Al iniciar el contacto, se registrará el tipo de etiqueta o registro dentro de 
etiqueta que se pretende ``despertar''. El tipo de etiqueta puede
definirse utilizando el formato \acs{MIME} (por ejemplo, \emph{app/mi\_tipo}).
De esta forma, la aplicación estará preparada para leer y escribir etiquetas
con este tipo.

El listado \ref{code:addListener} muestra un ejemplo en el que se añaden
dos \emph{Listener}s:

\begin{lstlisting}[
  language = Java,
  caption = {Se añade un \emph{Listener} para etiquetas \texttt{NDEF} y otro
  para etiquetas \texttt{NDEF} del tipo ``app/mi\_tipo''.},
  label = code:addListener]
  try{
    DiscoveryManager.getInstance().addTargetListener(this, TargetType.NDEF_TAG);
    DiscoveryManager.getInstance().addNDEFRecordListener(this, new NDEFRecordType(NDEFRecordType.MIME, ``app/mi_tipo''));
  } catch(ContactlessException e) { }
\end{lstlisting}

\item \textbf{Conexión y lectura/escritura de datos}. La  interfaz
\texttt{TargetListener} requiere la implementación del método
\texttt{targetDetected}, mediante el cual se definen las acciones a realizar 
cuando se establezca el contacto con una etiqueta \acs{RFID}. El listado
\ref{code:targetDetected} muestra un ejemplo de implementación:

\begin{lstlisting}[
  language = Java,
  caption = {Ejemplo de implementación del método \texttt{targetDetected} en
  la que se realizan varias operaciones.},
  label = code:targetDetected]
  public void targetDetected(TargetProperties[] prop) {
    NDEFTagConnection conn = null;
    //Obtenemos las propiedades de la etiqueta
    TargetProperties target = prop[0];
    try {
      // Obtenemos la url de la etiqueta para ``conectarnos'' con ella
      String url = target.getUrl();
      // Abrimos la conexion y la almacenamos en un objeto NDEFTagConnection
      conn = (NDEFTagConnection)Connector.open(url);
      // Leemos los datos de la etiqueta y los almacenamos en un objeto NDEFMessage
      NDEFMessage message = conn.readNDEF();
      // Obtenemos el numero de registros del mensaje
      NDEFRecord[] records = message.getRecords();
      // Operamos con esos registros y ejecutamos las acciones correspondientes
      // Podemos crear y escribir un nuevo mensaje en la etiqueta de esta forma
      conn.writeNDEF(message);
      //Cerramos la conexion
      conn.close();
    } catch (...) { }
  }
\end{lstlisting}

\item \textbf{Eliminación del \emph{Listener}}. Cuando ya no interese leer o 
escribir etiquetas de un tipo determinado, se eliminará su registro 
correspondiente. Con ello, la aplicación dejará de estar a la escucha de este 
tipo de etiquetas, a menos que se vuelvan a registrar (listado
\ref{code:removeListener}):

\begin{lstlisting}[
  language = Java,
  caption = {Se eliminan los \emph{Listener}s añadidos en el listado
  \ref{code:addListener}.},
  label = code:removeListener]
  try {
    DiscoveryManager.getInstance().removeNDEFRecordListener(this, new
    NDEFRecordType(NDEFRecordType.MIME,``app/mi_tipo''));
    DiscoveryManager.getInstance().removeTargetListener(this, TargetType.NDEF_TAG);
  } catch (...) { }
\end{lstlisting}

\item \textbf{Lectura instantánea de información}. El registro del
\texttt{NDEFRecordListener} permitirá leer datos con el simple
hecho de aproximar el dispositivo móvil a una etiqueta \acs{RFID}, sin 
necesidad de que, la aplicación que hará uso de sus datos, esté en ejecución.
Para ello, debe implementarse el método \emph{RecordDetected}.

El método \texttt{recordDetected} obtiene directamente el objeto de tipo
\texttt{NDEFMessage} que contiene el mensaje de la etiqueta \acs{RFID}
(figura \ref{code:recordDetected}).

\begin{lstlisting}[
  language = Java,
  caption = {El objeto \texttt{peli}, de tipo \texttt{Pelicula}, almacena
  el contenido del registro 0 de la etiqueta, extraído por el lector
  \acs{NFC}.},
  label = code:recordDetected]
  public void recordDetected(NDEFMessage nDEFMessage) {
    // Nos aseguramos que el mensaje contenido en el registro 0 de la etiqueta es de tipo ``app/objeto''
    if (nDEFMessage.getRecord(0).getRecordType().getName().equals(``app/pelicula'')){
      //Obtengo la informacion contenida en ese mensaje y construyo con ella un objeto especifico de mi aplicacion.
      peli = new Pelicula(nDEFMessage.getRecord(0).getPayload());
    }
  }
\end{lstlisting}

Una vez extraída la información de la etiqueta, comenzará la ejecución normal
de la aplicación, teniendo los datos de la etiqueta disponibles para su
tratamiento desde el principio. Sin embargo, para provocar el ``auto-arranque''
de la aplicación, es necesario dar de alta a la misma en el
\emph{Push-Registry} \texttt{Java} del dispositivo móvil.

\item \textbf{El \emph{Push-Registry}}. Es una clase interna \acs{JavaME} que 
permite ejecutar aplicaciones en principio inactivas, ante la respuesta a un 
evento, como puede ser el \emph{timer} de un reloj, la aproximación de una 
etiqueta \acs{RFID} al dispositivo, etc.

Cuando una aplicación define el \emph{Push-Registry} como respuesta a un
evento, este hecho queda registrado en el archivo description de la
aplicacion móvil (archivo con extensión \emph{.jad}).
\end{itemize}

